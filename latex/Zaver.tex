\chapter{Záver}

Cieľom projektu bolo navrhnúť a skonštruovať zariadenie na úpravu hlasitosti televízora, ktorý sme splnili, až na technický problém pri finálnom testovaní.

Pri postupe sme najprv navrhli po častiach jednotlivé funkcie kódu ktoré sme testovali na breadboarde. Jediná funkcia ktorá sa nám nepodarila realizovať bola predošle spomenutá funkcia na ukladanie kódov zmeny hlasitosti. Táto funkcia by iba vylepšila zariadenie a nebola kľúčová pre fungovanie, preto sme sa rozhodli ju vynechať. Keď sme mali pripravené všetky časti kódu, spojili sme ich do jedného programu a otestovali sme ho na prvom prototype – zapojenie na breadboarde. Po ďalších úpravách sme zostavili prototyp 2 aj s obalom vyrobeným v 3D tlačiarni.

Hlavný problém, s ktorým sme sa stretli pri tomto projekte bola obmedzená komunikácia a lockdown, kvôli ktorému bol znemožnený osobný kontakt a online komunikácia bola nedostačujúca.

Aj napriek týmto komplikáciám sme si overili schopnosť využitia získaných vedomostí počas tohto semestra, čo považujeme za významný osobný prínos.


